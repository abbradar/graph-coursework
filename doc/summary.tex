\documentclass[a4paper,12pt]{report}

\usepackage[utf8x]{inputenc}
\usepackage[english,russian]{babel}
\usepackage{ifpdf}
\usepackage{fancyhdr}

\ifpdf
  \usepackage{cmap}
\fi

\usepackage[T2A]{fontenc}

\title{Расчётно-пояснительная записка по курсовой "Машинная графика"}
\author{Николай Амиантов}
\date{\today}

\makeatletter
\let\thetitle\@title
\let\theauthor\@author
\let\thedate\@date
\makeatother

\renewcommand{\thesection}{\arabic{section}}
\setcounter{secnumdepth}{3}

\pagestyle{fancy}
\fancyhead{}
\fancyfoot{}
\fancyfoot[C]{\thepage}
\fancyfoot[R]{\thedate}

\begin{document}

% Поправить титульник чтобы был как по ГОСТу потом
\maketitle

\section{Введение}

Тема, её актуальность, цель и задачи работы.

\section{Анализ области}

\subsection{Отображение сцены}

\subsubsection{Хранение модели}

Во внешнем файле, почему так.

\subsubsection{Отрисовка модели}

\subsubsection{Учёт видимости модели}

\subsubsection{Освещение}

\subsection{Физика}

\subsubsection{Физическая модель}

Почему именно такая.

\subsubsection{Физика одного тела}

Гравитация, действия сил на точки.

\subsubsection{Физика нескольких тел}

Столкновения, их поиск и реакция.

\section{Технология}

Здесь описываем все алгоритмы которые выбрали, почему пишем на том-то и том-то и прочее.

% Пока не вижу смысла для исследовательского раздела, попробую поискать тему для исследования в ходе работы.

\section{Заключение}

Сделано то-то и то-то.

\end{document}