\documentclass[a4paper,12pt]{article}

\usepackage[utf8x]{inputenc}
\usepackage[english,russian]{babel}
\usepackage{ifpdf}
\usepackage{fancyhdr}

\ifpdf
  \usepackage{cmap}
\fi

\usepackage[T2A]{fontenc}

\title{Техническое задание по курсовой "Машинная графика"}
\author{Николай Амиантов}
\date{\today}

\makeatletter
\let\thetitle\@title
\let\theauthor\@author
\let\thedate\@date
\makeatother

\pagestyle{fancy}
\fancyhead{}
\fancyfoot{}
\fancyfoot[C]{\theauthor}
\fancyfoot[R]{\thedate}

\begin{document}

\section{Основание для разработки}

Программный продукт разрабатывается в рамках курсовой работы по дисциплине "Машинная графика".

\section{Тема разработки}

Моделирование детского трёхмерного конструктора.

\section{Функционал}
\begin{itemize}
  \item Добавление, удаление, перемешение объектов
  \begin{itemize}
    \item Основные объекты -- кубы, конусы, пирамиды
    \item Список объектов может быть дополнен извне с помощью файлов с описанием новых объектов
    \item Движение объектов -- перемещение, поворот
  \end{itemize}
  \item Трёхмерная сцена с плоскостью
  \begin{itemize}
    \item Собранное изделие располагается на виртуально бесконечной плоскости
  \end{itemize}
  \item Возможность передвижения камеры
  \begin{itemize}
    \item Вид от первого лица
  \end{itemize}
  \item Расположение объектов
  \begin{itemize}
    \item Объекты могут располагаться друг на друге, соединяться и разъединяться с учётом геометрических особенностей
  \end{itemize}
\end{itemize}

\end{document}