\documentclass[a4paper,12pt]{article}

\usepackage[utf8x]{inputenc}
\usepackage[english,russian]{babel}
\usepackage{ifpdf}
\usepackage{fancyhdr}

\ifpdf
  \usepackage{cmap}
\fi

\usepackage[T2A]{fontenc}

\title{Техническое задание по курсовой "Машинная графика"}
\author{Николай Амиантов}
\date{\today}

\pagestyle{fancy}
\fancyhead{}
\fancyfoot{}
\fancyfoot[C]{\thepage}
\fancyfoot[R]{\today}

\begin{document}

\section{Основание для разработки}

Программный продукт разрабатывается в рамках курсовой работы по дисциплине "Машинная графика".

\section{Назначение разработки}

Симуляция виртуальной физической трёхмерной песочницы с возможностью влияния на мир.

\section{Функционал}
\begin{itemize}
  \item Трёхмерная сцена с плоскостью
  \begin{itemize}
    \item Бесконечная плоскость
  \end{itemize}
  \item Возможность передвижения камеры
  \begin{itemize}
    \item Вид от первого лица
  \end{itemize}
  \item Добавление, удаление, перемешение объектов
  \begin{itemize}
    \item Объекты описываются в специальном формате вне программы
    \item Можно подгружать новые в программу
    \item Действия -- перемещение, поворот, масштабирование
  \end{itemize}
  \item Физика объектов
  \begin{itemize}
    \item Физика одного тела (гравитация, возможность воздействия на тело)
    \item Физика двух и более тел (столкновения)
  \end{itemize}
\end{itemize}

\end{document}